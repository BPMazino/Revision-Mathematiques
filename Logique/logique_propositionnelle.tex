\documentclass{article}
\usepackage[utf8]{inputenc}
\usepackage{amsthm}
\usepackage{minted}


\newtheorem{theorem}{Theorem}[section]
\newtheorem{corollary}{Corollary}[theorem]
\newtheorem{lemma}[theorem]{Lemma}

\theoremstyle{definition}
\newtheorem{definition}{Définition}[section]

\theoremstyle{remark}
\newtheorem*{remark}{Remarque}

\newtheorem*{notation}{Notation}


\title{Logique propositionnelle}
\author{Mazino }
\date{July 2022}



\begin{document}

\maketitle

\section{Introduction}

\section{Syntaxe}

\begin{definition}[Proposition]
On appelle \emph{proposition}, ou \emph{variable propositionnelle}, est un symbole qui a peut-être soit vrai soit faux. 
\end{definition}

\begin{definition}[Ensemble des propositions]
On appelle $E_P$ un ensemble infini ou fini et dénombrable de propositions. 
\end{definition}


\begin{definition}[Formule propositionnelle]
Informellement une \emph{formule propositionnelle} est une combinaison de \emph{connecteurs logiques} et de  \emph{variables propositionnelle}.\\
Plus formellement on peut définir une  \emph{formule propositionnelle} récursivement.\\
Soient $P$ une proposition, $\varphi, \psi$ deux formules.\\
Soit $P \in E_P$ une proposition. \\
$\neg \varphi$ est une formule.\\
$(\varphi \land \psi)$ et $(\varphi \lor \psi)$ sont des formules.
\end{definition}

\begin{definition}[Connecteur Logique]
On appelle \emph{connecteur logique} un symbole reliant deux formules.\\ 
On utilisera les éléments de l'ensemble suivant comme connecteur logique : $\{\neg, \land, \lor  \}$.\\
Respectivement symbole de \emph{négation}, de \emph{disjonction}, de \emph{conjonction}.
\end{definition}

\begin{remark}
La négation est un connecteur unaire et les autres sont des connecteurs binaires.
\end{remark}

\begin{remark}
A partir de ceux-ci on pourra définir d'autres connecteurs logiques tels que $\rightarrow$ ou $\leftrightarrow$.\\
Respectivement d'\emph{implication} et d'\emph{équivalence}.
\end{remark}


\begin{minted}{ocaml}
type formule = Proposition of string 
               | Not of formule
               | And of formule * formule 
               | Or of formule * formule
;;
\end{minted}

\end{document}
